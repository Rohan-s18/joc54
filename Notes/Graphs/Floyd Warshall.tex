\documentclass[12pt, letterpaper]{article}

\newcommand\tab[1][1cm]{\hspace*{#1}}

\title{Floyd Warshall Algorithm}
\author{Rohan Singh}
\date{April 21, 2023}

\begin{document}

\maketitle

\section{Introduction}
The \textbf{Floyd-Warshall Algorithm} is a dynamic-programming formulation to solve the all-pairs shortest-paths problem on a directed graph G = (V,E).

\section{Algorithm}
The basic idea behind the Floyd Warshall algorithm is to find the shortest path \textit{p} between any arbitrary indices i and j in the graph, by continously increasing the set of vertices that \textit{p} can have. We fill the table from having 0 intermediate vertices in \textit{p}, to including the set of all of the vertices in the graph in the path \textit{p}.\\\\
We can define the recurence relation in the Floyd-Warshall algorithm as:\\
$d^0_i_j$ = $w_i_j$\\
$d^k_i_j$ = min$\{$$d^k^-^1_i_j$, $d^k^-^1_i_k$ + $d^k^-^1_k_j$  $\}$ (k $\geq$ 0)\\
Here $d^k_i_j$ represents the shortest path between vertices i and j using the set of intermediate vertices $\{$1,2,3...k$\}$.\\\\
The pseudocode for the algorithm is:\\
\textbf{FLOYD-WARSHALL}(W, n):\\
$D^0$ = W\\
for k $\in$ (1...n):\\
\tab for i $\in$ (1...n):\\
\tab \tab for j $\in$ (1...n):\\
\tab \tab \tab $D^k_i_j$ = min$\{$$D^k^-^1_i_j$, $D^k^-^1_i_k$ + $D^k^-^1_k_j$  $\}$\\
return $D^n$


\section{Proof of Correctness}
To prove the correctness of the Floyd Warshall algorithm, we have to use induction.\\\\
\textbf{Base Case} (k=0):\\
When k = 0, we have 0 intermediate vertices in the path set in the shortest path between any 2 arbitrary vertices i and j. In this case the shortest path will just be the weight of the edge (if one exists) connecting the vertices i and j. Since the algorithm sets $D^0_i_j$ as $w_i_j$ the base case is true.\\\\
\textbf{Inductive Step}:\\
Let the algorithm be correct for k-1 iterations. This means that $D^k^-^1$ is the matrix that contains all of the weights of the shortest paths using the intermediate vertices $\{$1,2,3...k-1$\}$.\\\\
\textbf{Generalization}:\\
For the vertex k included in the set of intermediate vertices, there are to possible relations between the shortest path \textit{p} and the vertex k. Either k is not an intermediate vertex in \textit{p} or k is an intermediate vertex in the shortest path \textit{p}.\\\\
For the first case if the shortest path \textit{p} uses the vertices in the intermediate set $\{$1,2,3...k-1$\}$, then it also uses the vertices in the intermediate set $\{$1,2,3...k$\}$, since the later is the superset of the former.\\\\
In the second case, k is an intermediate vertex, hence we can find the shortest path will be the same as the shortest path between i and k plus the shortest path between k and j (as k is an intermediate vertex).\\\\
The following relation takes into account the 2 cases for k:\\
$D^k_i_j$ = min$\{$$D^k^-^1_i_j$, $D^k^-^1_i_k$ + $D^k^-^1_k_j$  $\}$\\\\
Since we assumed $D^k^-^1$ using the inductive step. Therefore, $D^k$ contains all of the shortest path weights using the set of intermediate vertices $\{$1,2,3...k$\}$\\\\
We have proved the Floyd Warshall algorithm to be correct using induction.


\section{Space and Runtime Complexity}
The runtime complexity of the Floyd-Warshall algorithm is $\Theta$($V^3$) because each of the for loops within the algorithm runs V times. Additionally, the space complexity of this algorithm is $\Theta$($V^3$) because we require a 3 dimensional data structure (Rank 3 Tensor) for the dynamic programming tabulation. 


\end{document}